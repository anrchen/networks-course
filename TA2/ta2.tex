\documentclass{llncs}

\usepackage{amsmath} % for equation*
\usepackage{color}
\usepackage{hyperref}
\usepackage{graphicx}
\definecolor{darkgreen}{rgb}{0,0.7,0}

% Fix link colors
\hypersetup{
    colorlinks = true,
    linkcolor=red,
    citecolor=red,
    urlcolor=blue,
    linktocpage % so that page numbers are clickable in toc
}

\newcounter{ques}
\setcounter{ques}{1}

\newcommand{\quest}[2]{\paragraph{}\textbf{Q\theques} - #1\stepcounter{ques} }

\newcommand{\answer}[1]{\color{red}\textit{#1}\color{black}}
\title{COMP 445 -- Theoretical Assignment 2 (TA2)}

\author{Tristan Glatard\\
  \href{mailto:tristan.glatard@concordia.ca}{tristan.glatard@concordia.ca}\\
  \vspace*{0.3cm}
  (adapted from Aiman Hanna)\\
  version 1.0
  }

\institute{Concordia University\\
  Department of Computer Science and Software Engineering}

\begin{document}

\maketitle

All questions will receive equal points. Please submit your assignment
as a pdf file on Moodle. The name of the pdf file must contain your
name and student id. Your name and student id must also appear in the
header of the pdf document. Please answer the questions in the order
used below and indicate the question number before your answer (e.g.,
\textbf{Q1}). Wherever possible, briefly indicate the method used to
obtain a numerical value, e.g., mathematical formula. Due date:
\textbf{Feb 17, 11:55pm}.

\section{Transport Layer}

\quest{Describe the one service provided by UDP that justifies the
  existence of this protocol in addition to network-level protocols
  such as IP. In addition, explain why applications may want to use UDP
  rather than TCP even though it provides less services.}

\answer{UDP provides \underline{multiplexing and de-multiplexing},
  i.e., it allows clients to communicate with specific
  \underline{processes} running on a server. IP, on the contrary, only
  enables communication among \underline{hosts}. UDP also provides a
  \underline{checksum} for the transmitted data while IP's checksum
  only covers the header of IP datagrams.  Applications may want to
  use UDP to \underline{avoid TCP's overhead} (connection
  establishment and retransmission) and \underline{throughput reduction}
  mechanisms (flow or congestion control).}

\quest{Explain why two TCP segments with identical destination ports
  and IP addresses may end up being delivered to two different sockets
  of this host. Can this situation happen with UDP and if yes, why?}

\answer{A TCP socket is defined as a \underline{4-tuple} (source address, source
  port, destination address, destination port). See slide 14 in
  Chapter 3's presentation. When an TCP segment is received by the
  transport layer, it is demultiplexed to a TCP socket based on these
  4 values. Thus, two segments with identical destination ports
  and IP addresses will be delivered to different sockets when their
  source addresses or ports are different.  This situation happens
  when 2 clients located on different hosts open TCP connections with
  the same process of the same server (e.g., they both open
  connections with the same Web server). Another way of seeing this is
  to keep in mind that a TCP connection has exactly 2 endpoints: a
  client socket and a server socket. Thus, when a client opens two
  TCP connections to the same server, each of these two connections
  has its own socket on the server. This situation cannot happen with
  UDP since UDP sockets are \underline{2-tuples} containing the destination port
  and address only (see slide 11).}


\quest{Based on your knowledge of the UDP header format, explain why
  UDP header field(s) impose a theoretical limit on the size of the
  data that can be carried in a UDP segment. What is this limit? Can
  this problem be addressed?}

\answer{The UDP header has a \underline{length field} containing the
  length of the UDP segment in bytes. Since this field is coded on 16
  bits, its maximal value is $2^{16}-1$=65,535. Therefore the maximal
  length of UDP segments is 65,535 bytes (including header and data)
  or about 65 KB. There is no way to address this problem without a
  revision of the UDP specification, e.g., by increasing the size of
  the length field. However, the UDP header field size is not a
  bottleneck with the current IP protocol (IPv4) since the maximal
  datagram size is also 65,535 bytes. That is, IPv4 could not carry
  larger UDP segments anyway.  In IPv6, so-called jumbograms can be up
  to $2^{32}-1$=4,294,967,295 bytes, i.e., more than 4GB. When UDP
  segments are carried in jumbograms, their length field must be set
  to 0.}

\quest{Explain how UDP segments can end up TCP segments being delayed,
  theoretically, indefinitely! If so, what would you propose as
  changes to UDP to mitigate this problem?  Your solution must mainly
  keep the advantages/purpose of UDP, while mitigating the problem at
  hand.}

\answer{Since \underline{UDP does not implement congestion control},
  clients using UDP could easily ``flood'' the network by transmitting
  segments at a high throughput regardless of network congestion. Such
  a behavior would result in high congestion leading to high packet
  loss and long queuing delays. This would dramatically impact TCP
  connections because TCP's congestion control would react to such
  high packet loss and long delays by reducing senders' sending rates,
  theoretically to 0. \underline{Congestion control must be added to
    UDP} to mitigate this problem. However, this must be done without
  adding reliability or in-order delivery to keep the advantages and
  purpose of UDP (adding reliability and in-order delivery would
  result in a TCP-like protocol with the associated overhead). The
  Datagram Congestion Control Protocol (DCCP, see RFC 4340) has been
  proposed as a congestion-control mechanism for UDP. RFC 5405 also
  proposes guidelines related to congestion control with UDP.  }

\quest{Show, through an example, how checksum could be inconclusive of
  error detection (i.e.  does not guarantee that errors can be
  detected). In your example, assume transmitted data is broken into
  24-bit chunks by the protocol utilizing checksum. In case errors are
  detected by checksum, does that fully (100\%) guarantee that errors
  must have actually occurred?}

\answer{Using the checksum algorithm used in UDP (see slide 10 in
  Chapter 3), we are looking for an example where A+B=C+D, A$\neq$C
  and B$\neq$D. The simplest example is obtained with C=B and D=A. For instance:
  \begin{itemize}
  \item A=00000000000000000000001
  \item B=00000000000000000000000
  \item C=00000000000000000000000
  \item D=00000000000000000000001
  \end{itemize}
  In this case, the checksum for (A,B) is the complement to 1 of
  (A+B), that is: 111111111111111111111110. And the checksum for (C,D)
  is the same value, although (A,B) and (C,D) are different data. This
  example, where two words in the initial data have been swapped, is
  not very likely to happen though. Other examples include:
  \begin{itemize}
  \item C=0 and D=A+B.
  \item One bit in A and the same bit in B are flipped.
  \item \ldots
  \end{itemize}
  This demonstrate that the checksum does not guarantee that errors
  can be detected, i.e., false negatives may occur. However, in case
  errors are detected by checksum, \underline{it is guaranteed that
    errors must have actually occurred}, even though it should be
  noted that errors might have occurred in the checksum field itself.
}

\quest{Explain why we sometimes need to add padding bytes to TCP
  segment headers.}

\answer{The TCP header length (or data offset) is specified in the TCP
  header in multiple of 32 bits (or in ``32-bit words''). That means
  that the TCP header size has to be a multiple of 32 bits. In case
  TCP options are defined with less than 32 bits, padding bytes are
  added to align the TCP header size with a 32-bit boundary.}

\quest{Explain the difference between congestion control and flow
  control. Mention the mechanisms in the TCP protocol that provide
  each of these two services (no description of the mechanisms is required).}

\answer{The goal of congestion control is to prevent \underline{the
    network} from being overwhelmed while the goal of flow control is
  to prevent \underline{the destination host} from being
  overwhelmed.  Mechanisms in TCP:
  \begin{itemize}
  \item Flow control: the receiver sends an explicit window size to
    the sender, the receiver window (\texttt{rwnd}), which corresponds
    to the amount of free buffer space in the receiver. The sender
    limits its rate by keeping the amount of ``in-flight'' data lower
    than \texttt{rwnd}.
  \item Congestion control: additive increase multiplicative decrease,
    congestion window (\texttt{cwnd}), slow start, congestion
    avoidance, fast recovery.
  \end{itemize}
}

\quest{TCP acknowledgments arrive after 26, 32 and 24ms. What is the
  new estimated RTT value after each of these acknowledgments
  arrived, taking an initial estimated RTT of 30ms and $\alpha=0.125$?}{1} 

\answer{According to slide 62 in Chapter 3's presentation:
  \begin{equation*}
    EstimatedRTT=(1-\alpha)*EstimatedRTT+\alpha SampleRTT
  \end{equation*}
  With $\alpha=0.125$ it gives:
  \begin{itemize}
  \item After the first ACK: \underline{EstimatedRTT=29.5ms}
  \item After the second ACK: \underline{EstimatedRTT=29.8ms}
  \item After the third ACK: \underline{EstimatedRTT=29.1ms}
  \end{itemize}
}

\quest{The TCP timer expires (i.e., a timeout occurs) while the
  congestion window size was 18KB, the congestion threshold was 20KB
  and TCP was in congestion avoidance mode. What will be the new
  window size after 4 successful transmissions? Assume a max segment
  size (MSS) of 1KB. Consider both TCP Reno and TCP Tahoe.}{1}

\answer{The behavior will be the same for Reno and Tahoe as these two
  versions of TCP only differ when triple duplicate ACKs are
  received. According to slide 104 in Chapter 3's presentation, when a
  timeout occurs:
  \begin{itemize}
  \item The system moves to the ``Slow Start'' mode.
  \item \texttt{ssthresh} is set to \texttt{cwnd}/2, i.e., \underline{\texttt{ssthresh}=9KB}.
  \item \texttt{cwnd} is set to 1 MSS, i.e. \underline{\texttt{cwnd}=1KB}.
  \item \texttt{dupACKcount} is set to 0.
  \end{itemize}
  In Slow Start mode, 1 MSS will be added to \texttt{cwnd} after each
  successful transmission, i.e., whenever an ACK is received. The
  system will remain in Slow Start mode as long as \texttt{cwnd} \textless
  \texttt{sstresh}, i.e., \texttt{cwnd} \textless 9KB. After 4 successful
  transmissions, i.e., after 4 ACKs are received,
  \underline{\texttt{cwnd} will be 5KB} and the system will still be
  in Slow Start mode.
}

\quest{Conduct a small research on SYN flooding and explain how this
  technique can be used as a DDoS attack on a TCP server. What are the
  efficient counter measures to such an attack?}{1}
\answer{
According to RFC 4987\footnote{\url{https://tools.ietf.org/html/rfc4987}}:
  \begin{quote}
    The SYN flooding attack is a \underline{denial-of-service} method
    affecting hosts that run TCP server processes.  The attack takes
    advantage of the state retention TCP performs for some time after
    receiving a SYN segment to a port that has been put into the
    LISTEN state.  The basic idea is to exploit this behavior by
    \underline{causing a host to retain enough state for bogus half-connections}
    that there are no resources left to establish new legitimate
    connections.
  \end{quote}
SYN flood attacks are conducted by sending a lot of SYN requests to
the target, and \underline{never acknowledging the corresponding
  SYN-ACK segments}.  The effective counter measures are
\underline{SYN caches and SYN cookies}. See RFC 4987 for more details.
}

\end{document}
