\documentclass{llncs}

\usepackage{amsmath} % for equation*
\usepackage{color}
\usepackage{hyperref}
\usepackage{graphicx}
\definecolor{darkgreen}{rgb}{0,0.7,0}

% Fix link colors
\hypersetup{
    colorlinks = true,
    linkcolor=red,
    citecolor=red,
    urlcolor=blue,
    linktocpage % so that page numbers are clickable in toc
}

\newcounter{ques}
\setcounter{ques}{1}

\newcommand{\quest}[2]{\paragraph{}\textbf{Q\theques} - #1\stepcounter{ques} }

\newcommand{\answer}[1]{\color{red}\textit{#1}\color{black}}
\title{COMP 445 -- Theoretical Assignment 1 (TA1)}

\author{Tristan Glatard\\
  \href{mailto:tristan.glatard@concordia.ca}{tristan.glatard@concordia.ca}\\
  \vspace*{0.3cm}
  (adapted from Aiman Hanna)\\
  version 1.0
  }

\institute{Concordia University\\
  Department of Computer Science and Software Engineering}

\begin{document}

\maketitle

All questions will receive equal points. Please submit your assignment
as a pdf file on Moodle. The name of the pdf file must contain your
name and student id. Your name and student id must also appear in the
header of the pdf document. Please answer the questions in the order
used below and indicate the question number before your answer (e.g.,
\textbf{Q1}). Wherever possible, briefly indicate the method used to
obtain a numerical value, e.g., mathematical formula. Due date:
\textbf{Feb 3, 11:55pm}.

\answer{Red text is used to report example solutions.}

\section{Introduction}

\quest{List two advantages and two drawbacks of organizing network protocols in layers.
  \quad}{10}

\answer{Advantages:
  \begin{itemize}
  \item Separation of concerns. Layers are assigned a specific and
    clear function. For instance, the application layer need not worry
    about how datagrams are routed in the network.
  \item Modularity. Layer implementation can be changed without any
    impact on other layers, as long as the service provided by the
    layer remains the same. For instance, link layer may rely on
    wireless or wired media with no impact on the higher
    layers. Modularity also makes the system easier to maintain.
  \item Clarity. Structuring a system in layers makes it easier to understand and simpler to
    implement.
  \end{itemize}
  Drawbacks:
  \begin{itemize}
    \item Redundancy. The same function may be re-implemented in
      different layers. For instance, data integrity (e.g., based on
      checksums) is provided by several layers.
    \item Rigidity. In the layered model, layer $n$ may only
      communication with layer $n-1$. In some cases it may be useful
      for a layer to have information about upper or lower layers.
    \item Overhead. Every layer adds its own header to the packets and
      requires some specific processing that increases the total overhead faced by the packets.
  \end{itemize}
}

\quest{Explain the difference between bandwidth and throughput. Are
  there situations where bandwith might be higher than throughput? And
  where throughput might be higher than bandwidth? If yes, provide
  examples. \quad}{10}

  \answer{ While bandwidth and throughput are both measured in bits
    per second, bandwidth is a property of a network link (also called
    \emph{capacity}) while throughput is the actual rate at which bits
    are transferred between network hosts. The bandwidth is the
    maximal throughput that can be achieved on a single network link,
    i.e., the throughput may not be higher than bandwidth. The
    throughput rarely reaches the bandwidth of a link. In general, the
    throughput remains lower than the bandwidth for the following
    reasons:
    \begin{itemize}
    \item The throughput is limited by the bandwidth of the slowest link on the source-destination path.
    \item The throughput might be intentionally limited by network protocols, e.g., to control congestion or avoid destination overload.
    \item The throughput might be intrinsically limited by network protocols, e.g., due to the need to establish connections.
    \item The throughput might be limited by network applications, e.g., depending on the type of subscription of their users.
    \item The throughput might be limited by the activity of other users in the network, when links are shared.
    \end{itemize}
    When data compression is used throughput measures may be higher
    than bandwidth but this should be considered as a measurement
    artifact.  }
  
\quest{Circuit Switching aims at providing a better service through the
reservation of the circuit (i.e. circuit is dedicated). Now,
considering only the perspective of the communicating users over a
Circuit Switching network (i.e. you should not be concerned with the
entire utilization of the network or the advantages to other users),
is it possible that Circuit Switching may actually end up harming its
users instead of providing a better service to them? If yes, provide a
scenario/case that shows that. If no, explain why this service will
indeed provide the best service to its users at all times.}{10}

\answer{In circuit switching:
  \begin{itemize}
  \item The throughput is limited by the number of circuits
      allocated.
    \item Connection setup and teardown adds overhead.
    \item Reliability depends on any single router in the circuit.
    \item It \emph{may} be easier to sniff information.
  \end{itemize}
}

\quest{Suppose two hosts, A and B, are d=10,000~km apart and are connected by a
  direct link of rate R=5~Mbps. Assume further that the propagation
  speed over the link is s=$2.5.10^8$m/s, and that the packets to be
  transferred are of size L=2~Mb.
  \begin{enumerate}
  \item What is the propagation delay to
    send one packet from A to B?
  \item What is the transmission delay to
    send 8 packets from A to B?
\end{enumerate}}{10}

\answer{
  Propagation delay:
  \begin{equation*}
    d_{prop}=\frac{d}{s}
  \end{equation*}
  It gives: $\underline{d_{prop}=0.04s}$.\\
  Transmission delay \underline{for 8 packets}:
  \begin{equation*}
    d_{trans} = 8.\frac{L}{R}
  \end{equation*}
  It gives: \underline{$d_{trans}= 3.2s$}.
}


\quest{Starting from the network described in the previous question, now assume that:
  \begin{itemize}
  \item A router X is installed at equal distance of A and B, that is, the network now looks like A -- X -- B.
  \item The link from A to X has rate $R_1$=4~Mbps.
  \item The link from X to B has $R_2$=6~Mbps.
  \item Processing and queuing delays in the router are negligible.
  \end{itemize}
  \begin{itemize}
  \item What is the propagation delay to send one packet from A to B?
  \item What is the total time needed to send 8 packets from A to B?
  \end{itemize}
}{10}

\answer{
  Propagation delay:
  \begin{equation*}
    d_{prop} =  \frac{d}{s}
  \end{equation*}
  Which gives: \underline{$d_{prop}=0.04s$} (it doesn't change).\\
  Total time: the total time to send \underline{the last packet} from A to X is:
  \begin{equation*}
    t_{A-X}=\frac{d_{prop}}{2}+8\frac{L}{R_1}
  \end{equation*}
  Since $R_2 \geq R_1$, the  first 7 packets have already been transmitted from X to B when the last packet arrives in X. The time to send the last packet from X to B is thus:
  \begin{equation*}
    t_{X-B}=\frac{d_{prop}}{2}+\frac{L}{R_2}.
  \end{equation*}
  Overall, the total time to send 8 packets from A to B is:
  \begin{equation*}
    t_{A-B}=t_{A-X}+t_{X-B}=d_{prop}+8\frac{L}{R_1}+\frac{L}{R_2}
  \end{equation*}
  It gives: \underline{$t_{A-B}=4.37s$}.
}

\quest{Perform a \texttt{traceroute} between your machine and any
  other host on the Internet, preferably overseas. Provide snapshots
  of what was returned and analyze the returned information. Comment
  on any behavior that looks unusual. Indicate the
  number of routers between your machine and the targeted
  host/server.}{10}

\answer{
  \begin{tiny}
\noindent  \$ traceroute en.hit.edu.cn \\
traceroute to en.hit.edu.cn (118.192.137.200), 30 hops max, 60 byte packets  \\
 1  buda-v536.encs.concordia.ca (132.205.221.129)  11.855 ms  11.778 ms  11.777 ms \\
 2  equity-v5.encs.concordia.ca (192.168.25.11)  0.232 ms  0.235 ms  0.244 ms \\
 3  buda-v5.encs.concordia.ca (192.168.25.9)  11.395 ms  11.403 ms  11.388 ms\\
 4  restraint.encs.concordia.ca (132.205.96.222)  0.322 ms  0.313 ms  0.301 ms\\
 5  gw-gojira-v014.concordia.ca (132.205.5.2)  0.802 ms  0.653 ms  0.660 ms\\
 6  10.99.100.53 (10.99.100.53)  1.131 ms  1.050 ms  0.934 ms\\
 7  10.98.1.34 (10.98.1.34)  0.807 ms  0.871 ms  0.873 ms\\
 8  gw-conu-lb8.concordia.ca (132.205.236.2)  2.183 ms  3.073 ms  3.299 ms\\
 9  concordia-internet.risq.net (206.167.255.17)  3.547 ms  3.304 ms *\\
10  concordia-internet.dmtrl-uq.risq.net (132.202.60.201)  3.360 ms  3.253 ms  3.682 ms\\
11  * imtrl-uq.risq.net (192.77.63.25)  2.658 ms  2.413 ms\\
12  ix-xe-1-0-1-500.tcore1.MTT-Montreal.as6453.net (206.82.135.33)  2.383 ms  2.027 ms  2.285 ms\\
13  if-ae-5-2.tcore1.W6C-Montreal.as6453.net (64.86.31.6)  81.456 ms  81.429 ms  81.541 ms\\
14  if-ae-3-2.tcore2.CT8-Chicago.as6453.net (66.198.96.46)  73.948 ms  73.345 ms  73.480 ms\\
15  if-ae-22-2.tcore1.CT8-Chicago.as6453.net (64.86.79.2)  74.126 ms  74.053 ms  73.660 ms\\
16  if-ae-29-2.tcore2.SQN-San-Jose.as6453.net (64.86.21.104)  74.282 ms  74.124 ms  74.148 ms\\
17  64.86.21.42 (64.86.21.42)  175.397 ms * *\\
18  202.97.50.61 (202.97.50.61)  177.997 ms  176.577 ms  176.499 ms\\
19  202.97.52.85 (202.97.52.85)  226.187 ms * *\\
20  202.97.58.113 (202.97.58.113)  316.092 ms *  308.232 ms\\
21  202.97.53.165 (202.97.53.165)  226.735 ms  224.463 ms  224.405 ms\\
22  * * *\\
23  158.234.120.106.static.bjtelecom.net (106.120.234.158)  223.293 ms * bj141-151-165.bjtelecom.net (219.141.151.165)  294.962 ms\\
24  * * *\\
25  * * *\\
26  * * *\\
27  121.101.208.190 (121.101.208.190)  247.692 ms  247.647 ms  291.890 ms\\
28  * * 121.101.208.210 (121.101.208.210)  325.099 ms\\
29  * * *\\
30  118.192.144.238 (118.192.144.238)  240.171 ms  240.188 ms  240.694 ms\\
\end{tiny}
  \begin{itemize}
  \item The IP address of the last listed router (\#30,
    118.192.144.238) is not the one of the target host
    (118.192.137.200). It means that 30 ``hops'' were not enough to
    reach the target host. The number of hops used by
    \texttt{traceroute} can be configured with option \texttt{-m}.
    \item Some hosts block the traceroute probes, which results in
      \texttt{*} being systematically displayed after a timeout (see,
      e.g., \#22 and \#24). Option \texttt{-T} can be used to bypass
      firewalls by making traceroute probes looks like regular TCP
      connection requests.
    \item \texttt{*} may also show occasionally, when responses are not received before the timeout. This may indicate packet loss or important delay. See, e.g., \#9 and \#17.
    \item Every time measure shown in the traceroute trace corresponds
      to the round-trip time of a specific packet. For instance, 6
      different packets were sent to obtain the 6 time measures in \#1
      and \#2. Delays may change between the sending of these different
      packets, which explains why timing may sometimes look
      inconsistent. For instance, the round-trip time in \#12 is lower
      than the one in \#10.
    \item The values in \#1 - \#8 look suspicious. First of all, the
      delays in \#1 and \#3 are consistently in the order of 11ms,
      which looks huge compared to other values. Second, \#2, \#3, \#6
      and \#7 show IP addresses in 192.168.0.0 or 10.0.0.0, which are
      private network addresses. Third, the route starts in the
      132.205.0.0 network (\#1), then goes to 192.168.0.0 (\#2 and
      \#3) before coming back to 132.205.0.0 (\#4 and \#5), going
      to 10.0.0.0 (\#6 and \#7) and back again to
      132.205.0.0 (\#8). It looks like a quite convoluted route!
  \end{itemize}
}

\quest{Conduct a small research on the ``Mirai'' malware and explain
  how it can be used to attack internet hosts.}{10}

\answer{Mirai is a worm that infects devices that still use default
  (factory) usernames and passwords. Devices such as home routers and
  IP cameras, the so-called Internet of Things (IoT), are particularly
  targeted. Mirai botnets were used to conduct massive DDoS attacks
  such as the one that targeted the Dyn DNS provider in October
  2016\footnote{\url{http://dyn.com/blog/dyn-analysis-summary-of-friday-october-21-attack/}}
  and had a significant impact on internet users in the East coast and
  beyond.}

%% see Yazbek's answer
%% see Zurka's answer
%% see Cloutier's answer

\section{Application Layer}

% HTTP and DNS
\quest{List the application and transport-layer protocol(s) involved in
  the retrieval of a Web page by a Web browser, assuming that the IP
  address of the Web server is initially unknown. Explain why these transport protocols are used.}{10}

\answer{
  \begin{itemize}
  \item Name-to-IP-address resolution:
    \begin{itemize}
    \item Application protocol: DNS
    \item Transport protocol: UDP (TCP may sometimes be used too).
    \end{itemize}
  \item Web page retrieval:
    \begin{itemize}
    \item Application protocol: HTTP
    \item Transport protocol: TCP
    \end{itemize}
  \end{itemize}
}

% email
\quest{Explain what happens when Alice sends an email to Bob and Bob's
  mail server is down}{10}

\answer{The message remains in Alice's mail server and waits for a new attempt.}

% P2P
\quest{(Textbook, Question P.22 – Page 179 (7 th ed.) / 177 (6 th
  ed.)). Consider distributing a file of size F=15~Gbits to N
  peers. The server has an upload rate of $u_s$ = 30 Mbps, and each peer
  has a download rate of $d_p$ = 2 Mbps and an upload rate of $u_p$. For N =
  10, 100, and 1000 and $u_p$ = 300 Kbps, 700 Kbps, and 2 Mbps, fill the
  following table by the minimum distribution time for each of the
  combinations of N and $u_p$ for both client-server distribution and P2P
  distribution.

  \setlength{\tabcolsep}{25pt}
  
  \textbf{Client Server}\\

  \begin{tabular}{c|c|c|c|c}
     && \multicolumn{3}{c}{N} \\
    &           & 10 & 100 & 1000 \\
    \hline
      & 300 kbps &  \hfill  &  \hfill    & \hfill      \\
    \cline{2-5}
   $u_p$& 700 kbps &    &     &      \\
   \cline{2-5}
     & 2 Mbps   &    &     &   
  \end{tabular}


  \textbf{Peer to Peer}\\

  \begin{tabular}{c|c|c|c|c}
     && \multicolumn{3}{c}{N} \\
    &           & 10 & 100 & 1000 \\
    \hline
      & 300 kbps &  \hfill  &  \hfill    & \hfill      \\
    \cline{2-5}
   $u_p$& 700 kbps &    &     &      \\
   \cline{2-5}
     & 2 Mbps   &    &     &   
  \end{tabular}
  }{10}

\setlength{\tabcolsep}{6pt}
\answer{
  \textbf{Client Server}\\
  The time required to transfer F bits to N clients in a client-server architecture is:
  \begin{equation}
    D_{CS} = \max \left( \frac{NF}{u_s}, \frac{F}{d_p} \right)
  \end{equation}
  (In seconds)
  \begin{tabular}{c|c|c|c|c}
     && \multicolumn{3}{c}{N} \\
     &             & 10      & 100        & 1000 \\
    \hline
      & 300 kbps   &  7,500  &  50,000    & 500,000      \\
    \cline{2-5}
   $u_p$& 700 kbps &  7,500  & 50,000     & 500,000     \\
   \cline{2-5}
     & 2 Mbps      &  7,500  &  50,000    & 500,000
  \end{tabular}\\
  \textbf{Peer to Peer}\\
    The time required to transfer F bits to N peers in a P2P architecture is:
  \begin{equation*}
    D_{P2P} \geq \max \left( \frac{F}{u_s},\frac{F}{d_p},\frac{NF}{u_s+N.u_p} \right)
  \end{equation*}
  (In seconds)
  \begin{tabular}{c|c|c|c|c}
     && \multicolumn{3}{c}{N} \\
    &              & 10      & 100      & 1000 \\
    \hline
      & 300 kbps   & 7,500   &  25,000  & 45,454      \\
    \cline{2-5}
   $u_p$& 700 kbps & 7,500   &  15,000  & 20,547     \\
   \cline{2-5}
     & 2 Mbps      & 7,500   &  7,500   & 7,500
  \end{tabular}  
}

\end{document}
