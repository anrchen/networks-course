\documentclass{llncs}

\usepackage{amsmath} % for equation*
\usepackage{wasysym} % for \Box
\usepackage{color}
\usepackage{hyperref}
\usepackage{graphicx}
\definecolor{darkgreen}{rgb}{0,0.7,0}

\newcommand{\myspace}[0]{\vspace*{0.25cm}}

% Fix link colors
\hypersetup{
  colorlinks = true,
  linkcolor=red,
  citecolor=red,
  urlcolor=blue,
  linktocpage % so that page numbers are clickable in toc
}


\newcommand{\answer}[1]{\color{red}\textit{#1}\color{black}}
\title{COMP 445 -- Theoretical Assignment 2 (TA1)\\ Winter 2018}

\author{Tristan Glatard\\
  \href{mailto:tristan.glatard@concordia.ca}{tristan.glatard@concordia.ca}
}

\institute{Concordia University\\
  Department of Computer Science and Software Engineering}

\begin{document}

\maketitle

\section*{Instructions}

\begin{itemize}
\item Please submit your assignment  as a pdf file on Moodle. The name of the pdf file must contain your name and student id. 
\item All questions will receive equal points.
\item Each question may have zero, one, or more than one
  correct choices.
\item Partial answers will not
  receive any point.
\item Blank answers (no answer) will not be penalized.
\end{itemize}

\hrulefill\\

\myspace

\myspace

Student ID: \dotfill

\myspace

\myspace

First Name / Last Name: \dotfill

\myspace

\myspace

Signature: \dotfill

\myspace

\myspace

\hrulefill

\newpage

\section*{Transport Layer}

\paragraph{\textbf{Q1:}}
Among the following services, which ones are provided by UDP?

\begin{tabular}{ccl}
  a) & $\Box$ &  Congestion control\\
  \\
  b) & $\Box$ &  Flow control \\
  \\
  c) & $\Box$ &  Reliable data transfer\\
  \\
  d) & $\Box$ &  Bandwidth reservation
\end{tabular}

\answer{None of these services are provided by UDP. See slide 16.}

% An application should use UDP vs plain IP (network layer protocol) because...
\paragraph{\textbf{Q2:}}
An application may choose to transmit data using UDP rather than just IP (the network-level protocol) when it needs:

\begin{tabular}{ccl}
  a) & $\Box$ &  High throughput\\
  \\
  b) & $\CheckedBox$ &  Multiplexing / de-multiplexing\\
  \\
  c) & $\Box$ &  Security\\
  \\
  d) & $\Box$ &  Connection management (establishment, teardown)
\end{tabular}

\answer{Multiplexing and de-multiplexing of datagrams through sockets
  is the main service provided by UDP, through port numbers. UDP
  doesn't provide security or connection management. UDP would provide
  a high throughput \emph{compared to TCP} but not compared to only IP.}

\paragraph{\textbf{Q3:}}
% TCP RTT calculation
TCP acknowledgments arrive with RTT values of 29, 31 and 32ms. What is the new
estimated RTT value after the third acknowledgement was received, taking an
initial estimated RTT of 31ms and $\alpha$ = 0.125?

\begin{tabular}{ccl}
  a) & $\Box$ &  32ms\\
  \\
  b) & $\CheckedBox$ &  30.9ms\\
  \\
  c) & $\Box$ &  31.8ms\\
  \\
  d) & $\Box$ &  31.1ms
\end{tabular}

\answer{See slide 62 in Chapter 3.
  $\mathrm{EstimatedRTT} = (1-\alpha)\mathrm{EstimatedRTT} + \alpha\mathrm{SampleRTT}$.
  \begin{itemize}
  \item Before the first ACK is received: $\mathrm{EstimatedRTT}$=31ms.
  \item After the first ACK is received: $\mathrm{EstimatedRTT}$=(1-0.125)*31+0.125*29=30.75ms
  \item After the second ACK is received: $\mathrm{EstimatedRTT}$=(1-0.125)*30.75+0.125*31=30.78ms
  \item After the third ACK is received: $\mathrm{EstimatedRTT}$=(1-0.125)*30.78+0.125*32=30.93ms
  \end{itemize}
}

\paragraph{\textbf{Q4:}}
Two non-duplicate ACKs are received while a TCP sender is in Slow
Start mode with cwnd=1KB, ssthresh=64KB and MSS=1KB. What is the
state of the TCP sender after the second ACK is received?

\begin{tabular}{ccl}
  a) & $\CheckedBox$ &  Slow Start\\
  \\
  b) & $\Box$ &  Congestion Avoidance\\
  \\
  c) & $\Box$ & Fast Recovery \\
  \\
  d) & $\Box$ & SYN sent \\
\end{tabular}

\answer{See slide 101 in Chapter 3. In Slow Start mode, 1MSS is added
  to cwnd every time a non-duplicate ACK is received. In addition,
  state changes to Congestion Avoidance when cwnd reaches the
  slow-start threshold (ssthresh). In our situation:
  \begin{itemize}
  \item After the first ACK is received: cwnd=2KB; cwnd \textless ssthresh; state remains Slow Start.
  \item After the second ACK is received: cwnd=3KB; cwnd \textless sstresh; state remains Slow Start.
\end{itemize}}

\paragraph{\textbf{Q5:}}
Assuming the same initial state and sequence of events as in the
previous question, what will be the value of the cwnd variable (size
of the congestion window) after the second ACK is received?

\begin{tabular}{ccl}
  a) & $\Box$ &  1KB\\
  \\
  b) & $\Box$ &  2KB\\
  \\
  c) & $\CheckedBox$ &  3KB\\
  \\
  d) & $\Box$ &  4KB
\end{tabular}

\answer{See answer to the previous question.}

% TCP/UDP trace analysis
\paragraph{\textbf{Q6:}}
The content below was captured using
Wireshark:\\
\includegraphics[width=\textwidth]{tcp.png}
This trace contains:\\

\begin{tabular}{ccl}
  a) & $\CheckedBox$ & A TCP segment that contains an acknowledgment  \\
  \\
  b) & $\Box$ & A UDP segment that contains an acknowledgment \\
  \\
  c) & $\Box$ & A TCP segment that contains an HTTP message \\
  \\
  d) & $\Box$ & An HTTP message that contains a TCP segment \\
\end{tabular}

\answer{The trace shows a TCP segment (see blue highlight). The
  segment contains an acknowledgment, as shown by the flags. b) The
  trace doesn't show any UDP segment (and it couldn't, as a packet
  couldn't use both UDP and TCP). c) Although a TCP segment might
  contain an HTTP message, it is not the case here: the length of the
  segment is 0, which means that the segment doesn't contain any data
  (it's just an ACK). d) An HTTP message may not contain a TCP segment
  since HTTP is an application-level protocol and TCP is transport-level.}

% Mechanisms to reliable data transfers
%   checksum, timeouts, acks, sequence numbers
\paragraph{\textbf{Q7:}}
Among the following mechanisms, which one(s) can be used to provide reliable data transfer? 

\begin{tabular}{ccl}
  a) & $\CheckedBox$ & Checksums  \\
  \\
  b) & $\CheckedBox$ &  Timeouts\\
  \\
  c) & $\CheckedBox$ &  Sequence numbers\\
  \\
  d) & $\CheckedBox$ &  Acknowledgments\\
\end{tabular}

\answer{All these mechanisms may be used to provide reliable data transfer. See slides 28-41.}

\paragraph{\textbf{Q8:}}
In connection-oriented multiplexing, two packets with the same
 source host, source port, destination host and destination port
 might be delivered to two different sockets:
 
\begin{tabular}{ccl}
  a) & $\Box$ & Yes\\
  \\
  b) & $\CheckedBox$ &  No\\
\end{tabular}

\answer{In connection-oriented (de-)multiplexing, packets are assigned
  to sockets based on the source host, source port, destination host and destination port only. See slide 12. }

\paragraph{\textbf{Q9:}}
What is the UDP checksum of D=10101010101010101010101010101010?

\begin{tabular}{ccl}
  a) & $\CheckedBox$ & 1010101010101010\\
  \\
  b) & $\Box$ & 0101010101010101\\
  \\
  c) & $\Box$ & 1111111111111111\\
  \\
  d) & $\Box$ & 0000000000000000\\
\end{tabular}

\answer{See slide 19. The checksum is the complement to 1 of the sum of each 16-bit word in D:\\
\begin{tabular}{l|rl}
 word 1&&1010101010101010\\
 word 2&+&1010101010101010\\
\hline
wraparound&1&0101010101010100\\
sum&&0101010101010101\\
\hline
checksum&&1010101010101010
\end{tabular}
}

\paragraph{\textbf{Q10:}}
In a pipelined protocol, the sender allows N simultaneous non-acknowledged packets. This is meant to:

\begin{tabular}{ccl}
  a) & $\Box$ & Increase network utilization, by a factor of $\frac{N}{2}$.\\
  \\
  b) & $\CheckedBox$ &  Increase network utilization, by a factor of $N$.\\
  \\
  c) & $\Box$ &  Reduce packet queuing time, by a factor of $\frac{N}{2}$.\\
  \\
  d) & $\Box$ &  Reduce packet queuing time, by a factor of $N$.\\
\end{tabular}

\answer{Pipelining increases network utilization by a factor of $N$, see slide 45. It won't reduce packet queuing time -- it might even increase it when there is congestion.}



\end{document}
