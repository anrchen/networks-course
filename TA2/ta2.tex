\documentclass{llncs}

\usepackage{amsmath} % for equation*
\usepackage{color}
\usepackage{hyperref}
\usepackage{graphicx}
\definecolor{darkgreen}{rgb}{0,0.7,0}

% Fix link colors
\hypersetup{
    colorlinks = true,
    linkcolor=red,
    citecolor=red,
    urlcolor=blue,
    linktocpage % so that page numbers are clickable in toc
}

\newcounter{ques}
\setcounter{ques}{1}

\newcommand{\quest}[2]{\paragraph{}\textbf{Q\theques} - #1\stepcounter{ques} }

\newcommand{\answer}[1]{}%\color{red}\textit{#1}\color{black}}
\title{COMP 445 -- Theoretical Assignment 2 (TA2)}

\author{Tristan Glatard\\
  \href{mailto:tristan.glatard@concordia.ca}{tristan.glatard@concordia.ca}\\
  \vspace*{0.3cm}
  (adapted from Aiman Hanna)\\
  version 1.0
  }

\institute{Concordia University\\
  Department of Computer Science and Software Engineering}

\begin{document}

\maketitle

All questions will receive equal points. Please submit your assignment
as a pdf file on Moodle. The name of the pdf file must contain your
name and student id. Your name and student id must also appear in the
header of the pdf document. Please answer the questions in the order
used below and indicate the question number before your answer (e.g.,
\textbf{Q1}). Wherever possible, briefly indicate the method used to
obtain a numerical value, e.g., mathematical formula. Due date:
\textbf{Feb 17, 11:55pm}.

\section{Transport Layer}

% Multiplexing - demultiplexing

\quest{Describe the one service provided by UDP that justifies the
  existence of this protocol in addition to network-level protocols
  such as IP. In addition, explain why applications may want to use UDP
  rather than TCP even though it provides less services.}

\answer{UDP provides multiplexing and de-multiplexing, i.e., it allows
  clients to communicate with specific processes on the server. IP, on
  the contrary, only enables communication among hosts.  Applications
  may want to use UDP to avoid TCP's overhead (e.g., connection
  establishment, error recovery) and throughput reduction mechanisms
  (for flow control or congestion control).}

\quest{Explain why two TCP segments with identical destination ports
  and IP addresses may end up being delivered to two different sockets
  of this host. Can this situation happen with UDP and if yes, why?}

\answer{This situation happens when 2 clients located on different
  hosts open TCP connections with the same process of the same server
  (e.g., they both open connections with the same Web server). See slide 14. This situation cannot happen with UDP (see slide 11).}

% UDP

\quest{Based on your knowledge of the UDP header format, explain why
  UDP header field(s) impose a theoretical limit on the size of the
  data that can be carried in a UDP segment. What is this limit? Can
  this problem be addressed?}

\answer{The UDP header has a \texttt{length} field coded on 16
  bits. This field contains the length of the segment in bytes. Its
  maximal value is $2^{16}-1$=65,535. Therefore the maximal length of
  UDP segments is 65,535 bits. There is no way to address this problem
  without a revision of the UDP specification. However, the UDP header
  field size is not a bottleneck with IPv4 since the IPv4 maximal
  datagram size is also 65,535 bits.}

\quest{Explain how UDP segments can end up TCP segments being delayed,
  theoretically, indefinitely! If so, what would you propose as
  changes to UDP to mitigate this problem?  Your solution must mainly
  keep the advantages/purpose of UDP, while mitigating the problem at
  hand.}

\answer{Since UDP doesn't implement congestion control, flooding the
  network with UDP segments would result in high congestion leading to
  high packet loss and long queuing delays. If packet loss is high,
  the congestion window will remain low, around 1 segment, and TCP
  will remain in slow-start mode. As the packet loss rate increases,
  the probability that a client receives an ACK decreases (receiving
  an ACK requires 2 segments to be successfully transmitted, the data
  segment and the ACK). Meanwhile, if queuing delays are high, the TCP
  timeout value will grow, which will further delay retransmission. }

\quest{Show, through an example, how checksum could be inconclusive of
  error detection (i.e.  does not guarantee that errors can be
  detected). In your example, assume transmitted data is broken into
  24-bit chunks by the protocol utilizing checksum. In case errors are
  detected by checksum, does that fully (100\%) guarantee that errors
  must have actually occurred?}


% TCP

\quest{Explain why we sometimes need to add padding bytes to TCP
  segment headers.}

\answer{Because the header size is expressed in 32-bit words.}

\quest{Explain the difference between congestion control and flow
  control. Mention the mechanisms in the TCP protocol that provide
  each of these two services (no description of the mechanisms is required).}

%\quest{Summarize the mechanisms that make TCP a reliable data transfer
%  protocol.}

% \quest{To what extent does Network Address Translation (NAT) break the
%  Internet protocol stack model? List at least two reasons.}

\quest{TCP acknowledgments arrive after 26, 32 and 24ms. What is the
  new estimated RTT value after each of these acknowledgments
  arrived, taking an initial estimated RTT of 30ms and $\alpha=0.125$?}{1} 

\answer{See slide 62}

\quest{The TCP timer expires (i.e., a timeout occurs) while the
  congestion window size was 18KB, the congestion threshold was 20KB
  and TCP was in congestion avoidance mode. What will be the new
  window size after 4 successful transmissions? Assume a max segment
  size (MSS) of 1KB. Consider both TCP Reno and TCP Tahoe.}

\answer{See slide 104.}

\quest{Conduct a small research on SYN flooding and explain how this
  technique can be used as a DDos attack on a TCP server. What are the
  efficient counter measures to such an attack?}

\answer{See RFC 4987.}

\end{document}
