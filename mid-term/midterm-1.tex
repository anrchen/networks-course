\documentclass{llncs}

\usepackage{amsmath} % for equation*
\usepackage{color}
\usepackage{hyperref}
\usepackage{graphicx}
\definecolor{darkgreen}{rgb}{0,0.7,0}

% Fix link colors
\hypersetup{
    colorlinks = true,
    linkcolor=red,
    citecolor=red,
    urlcolor=blue,
    linktocpage % so that page numbers are clickable in toc
}

\pagestyle{plain}

\newcounter{ques}
\setcounter{ques}{1}

\newcommand{\todo}[1]{\color{blue}\textbf{TODO:} #1\color{black}}
\newcommand{\myspace}[0]{\vspace*{0.25cm}}

\renewcommand{\question}[1]{\paragraph{}\textbf{Q\theques} - #1\stepcounter{ques} }

\newcommand{\answer}[1]{\color{red}\textit{#1}\color{black}}
\title{COMP 445 -- Midterm exam \\ Make-up version \\ Winter 2017 \\ Wednesday, March 16, 2017 \\ Duration: 75 minutes}

\author{Tristan Glatard\\
  \href{mailto:tristan.glatard@concordia.ca}{tristan.glatard@concordia.ca}\\
  \vspace*{0.3cm}
  }

\institute{Concordia University\\
  Department of Computer Science and Software Engineering}

\begin{document}

\maketitle

\section*{Instructions}
\begin{itemize}
\item All questions will receive equal points.
\item Answer all questions on these sheets in the space provided.
\item No books, notes or extra paper.
\item No cell phones, laptops or any electronic devices except ENCS calculators.
\item This exam is 11 pages long, including the cover page. It has 10 questions labeled from \textbf{Q1} to \textbf{Q10} (one question per page except on this cover page). Check that your copy is complete.
\item This exam counts for 20\% of your final grade.
\end{itemize}

\myspace

\myspace

\hrulefill\\

\myspace

Consistent with the university regulations concerning cheating and plagiarism I will not cheat during this examination:

\myspace

\myspace

Student ID: \dotfill

\myspace

\myspace

First Name / Last Name: \dotfill

\myspace

\myspace

Signature: \dotfill

\myspace

\myspace

\hrulefill

\newpage

%%%% Introduction

\question{Describe the main differences between packet-switched and circuit-switched networks.}

\answer{}

\newpage

\question{Cork (Ireland) and Lannion (France) are
  connected by a submarine cable of length l=565km. What is the
  total delay to transmit N=7 packets of size L=65,000 bits from
  Cork to Lannion through this cable, assuming that the cable bandwidth is
  R=40Gbps and its propagation speed is s=2.5.$10^8$m/s? Only
  transmission and propagation delays will be considered.}

\answer{ According to slide 45 to 47 in Chapter 1, the delay $d$ to
  transmit N packets will be the delay to transmit the last packet, i.e.:
  \begin{equation*}
    d=Nd_{trans} + d_{prop} \\
  \end{equation*}
  Where $d_{trans}$ and $d_{prop}$ are the transmission and
  propagation delays of a single packet, respectively:
  \begin{equation*}
    d_{trans} = \frac{L}{R} \quad \mathrm{and} \quad d_{prop} = \frac{l}{s}
  \end{equation*}
  It gives:
  \begin{equation*}
    d = 7\frac{65,000}{40.10^9}+\frac{565.10^3}{2.5.10^8}
  \end{equation*}
  \begin{equation*}
    d = 2.26 ms
  \end{equation*}
}

\newpage
%% \question{What is the value of N beyond which the transmission delay is higher than the propagation delay?}

%% \answer{
%%   \begin{equation*}
%%   N\frac{L}{R} \geq \frac{d}{s} \\
%%   N \geq \frac{d.R}{s.L}
%%   \end{equation*}

%%   It gives:
%%   \begin{equation*}
%%     N \geq \frac{4,600.10^3.40.10^9}{2.5.10^8.65,000}\\
%%     N \geq 11,323
%%   \end{equation*}
%% }

\question{List two kinds of security attacks that may be used against computer networks.}
\answer{DoS, IP spoofing, malware, packet sniffing.}

\newpage

\question{Consider a network link shared by N=9 users, each user
  being active 10\% of the time (the probability that a given user is
  active at any given time is $p=0.1$). What is the probability $p_2$ that exactly $2$
  users are active simultaneously?}

\answer{See slide 30 in Chapter 1.
  The probability that exactly $i$ users are active simultaneously is:
  \begin{equation*}
    p_i = \binom{N}{i} p^i(1-p)^{(N-i)}
  \end{equation*}
  It gives:
  \begin{equation*}
    p_2 = 36.0.1^2.(1-0.1)^{9-2}
  \end{equation*}
  \begin{equation*}
    p_2 = 0.17
  \end{equation*}
}

\newpage

%\quest{What are the different sources of packet  \underline{delay} in a packet-switching network?}

% \quest{End-to-end throughput on the network of Figure 1?}

% \quest{Encapsulation and how it relates to the layer model}

% Loss, delay, throughput

% Protocol layers

% Security

%%%% Application Layer

% Principles

% Web and HTTP

\question{The content between the two lines below was captured using
  Wireshark:
  
\noindent \hrulefill\\
\begin{small}
    HTTP/1.1 \textbf{200} OK\\
    Cache-Control: max-age=86400\\
    Content-Type: text/css\\
    Content-Encoding: gzip\\
    \textbf{Last-Modified}: Mon, 30 Jan 2017 15:37:21 GMT\\
    Accept-Ranges: bytes\\
    ETag: "aafd10bfe7bd21:0"\\
    Vary: Accept-Encoding\\
    Server: Microsoft-IIS/7.5\\
    X-Powered-By: ASP.NET\\
    Date: Thu, 16 Mar 2017 20:58:39 GMT\\
    Content-Length: 2517\\
    \textbf{\textbackslash r \textbackslash n}\\
\vspace*{-0.3cm}
\hrulefill

\end{small}
\begin{enumerate}
\item To which protocol does this content correspond to?
\item To which layer in the Internet model does this protocol belong to?
\item What is the meaning of the strings in bold face (\textbf{200}, \textbf{Last-Modified} and \textbf{\textbackslash r \textbackslash n})? 
\item Is this content coming from a client or from a server?
\item Describe a possible follow-up message for this content, i.e.,
  how the client or server may respond to this message. Only a
  high-level description of the message is expected, i.e., you don't
  have to write the complete message explicitly.
\end{enumerate}
}

\newpage

\answer{See slide 27 in Chapter 2. 
  \begin{enumerate}
  \item This trace is from the HTTP protocol.
  \item It belongs to the Application Layer.
  \item 
  \item From a server, this is an HTTP response.
  \item The client might issue another HTTP request to fetch the objects in this web page.
  \end{enumerate}
}

% \quest{Explain the interest of  using Web caching.}

% DNS

\question{What is the purpose of using a Web cache? Explain the interest through a simple example.}

%% \answer{See slide 62, 67 and 69 in Chapter 2. The main purpose of DNS is to resolve IP addresses from host
%%   names. DNS is a distributed database. In a recursive query, the
%%   query is forwarded from the client performing the query to the
%%   various servers involved in the response. In an iterated query, the
%%   client itself issues successive queries to the servers involved in the response.}

\newpage

% P2P

%\quest{Explain the formula on slide 79.}

% Socket programming

% coding mistake.

%%%% Transport Layer

\question{What is a port number? Give an example for a protocol of
  your choice. What is it used for in the transport layer? }

% Multiplexing and demux

% UDP

\newpage

% Reliable data transfer

%\question{Explain the difference between stop-and-wait and pipelined protocols. Quantify the performance gain provided by pipelined protocols. Give examples of such protocols.}

% Flow control

\question{Explain the purpose, meaning and use of the ACK flag
  in the TCP segment header.}


\newpage

% Congestion control

\question{Explain the difference between network-assisted and end-to-end
  congestion control. Which one of these approaches is used by TCP? And by UDP?}

%% \answer{See slide 95 in Chapter 3. In network-assisted congestion
%%   control, the routers are responsible for congestion detection and
%%   send feedback to the senders. In end-to-end congestion control, the
%%   senders themselves infer congestion status based on packet losses
%%   and delays, without receiving any explicit feedback from the
%%   network. TCP adopts an end-to-end approach to congestion
%%   control. UDP doesn't have any congestion control mechanism.}

\newpage

% TCP

%\question{Explain the concept of cumulative ACKs.}

%\quest{Why is the ACK bit needed in the TCP header since there is already a field for ack number?}

%\quest{Show a TCP time diagram: what should be the ACK value returned by the server?}

% \question{Explain what the cwnd variable is (no description of how it evolves is expected).}

\question{Two non-duplicate ACKs are received while a TCP sender is in Congestion Avoidance mode with \texttt{cwnd}=64KB, \texttt{ssthresh}=64KB and \texttt{MSS}=1KB. Describe the state of the TCP sender after reception of each of these two ACKs, more precisely:
  \begin{itemize}
  \item The state in which the TCP sender is (e.g., Congestion Avoidance).
  \item The value of the \texttt{cwnd} variable.
  \item The value of the \texttt{ssthresh} variable.
  \end{itemize}
}

%\answer{See slide 104 in Chapter 3. \texttt{cwnd} is increased by MSS*MSS/cwnd after each ACK is received. \texttt{ssthresh} doesn't change. TCP remains in Congestion Avoidance mode.}

\end{document}
