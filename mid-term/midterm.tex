\documentclass{llncs}

\usepackage{amsmath} % for equation*
\usepackage{color}
\usepackage{hyperref}
\usepackage{graphicx}
\definecolor{darkgreen}{rgb}{0,0.7,0}

% Fix link colors
\hypersetup{
    colorlinks = true,
    linkcolor=red,
    citecolor=red,
    urlcolor=blue,
    linktocpage % so that page numbers are clickable in toc
}

\pagestyle{plain}

\newcounter{ques}
\setcounter{ques}{1}

\newcommand{\todo}[1]{\color{blue}\textbf{TODO:} #1\color{black}}
\newcommand{\myspace}[0]{\vspace*{0.25cm}}

\renewcommand{\question}[1]{\paragraph{}\textbf{Q\theques} - #1\stepcounter{ques} }

\newcommand{\answer}[1]{\color{red}\textit{#1}\color{black}}
\title{COMP 445 -- Midterm exam \\ Winter 2017 \\ Wednesday, March 1, 2017 \\ Duration: 75 minutes}

\author{Tristan Glatard\\
  \href{mailto:tristan.glatard@concordia.ca}{tristan.glatard@concordia.ca}\\
  \vspace*{0.3cm}
  }

\institute{Concordia University\\
  Department of Computer Science and Software Engineering}

\begin{document}

\maketitle

\section*{Instructions}
\begin{itemize}
\item All questions will receive equal points.
\item Answer all questions on these sheets in the space provided.
\item No books, notes or extra paper.
\item No cell phones, laptops or any electronic devices except ENCS calculators.
\item This exam is 11 pages long, including the cover page. It has 10 questions labeled from \textbf{Q1} to \textbf{Q10} (one question per page except on this cover page). Check that your copy is complete.
\item This exam counts for 20\% of your final grade.
\end{itemize}

\myspace

\myspace

\hrulefill\\

\myspace

Consistent with the university regulations concerning cheating and plagiarism I will not cheat during this examination:

\myspace

\myspace

Student ID: \dotfill

\myspace

\myspace

First Name / Last Name: \dotfill

\myspace

\myspace

Signature: \dotfill

\myspace

\myspace

\hrulefill

\newpage

%%%% Introduction

\question{Explain what is the main cause of packet loss in a packet-switched network?}

\answer{See slide 43 in Chapter 1. Packets arriving at routers are
  dropped (lost) when there is no free space in routers' buffers. In
  other words, the main cause of packet loss are buffer overflows in
  routers.}

\newpage

\question{Halifax (NS, Canada) and Brean (United Kingdom) are
  connected by a submarine cable of length l=4,600km. What is the
  total delay to transmit N=12 packets of size L=65,000 bits from
  Halifax to Brean through this cable, assuming that the cable bandwidth is
  R=40Gbps and its propagation speed is s=2.5.$10^8$m/s? Only
  transmission and propagation delays will be considered.}

\answer{ According to slide 45 to 47 in Chapter 1, the delay $d$ to
  transmit N packets will be the delay to transmit the last packet, i.e.:
  \begin{equation*}
    d=Nd_{trans} + d_{prop} \\
  \end{equation*}
  Where $d_{trans}$ and $d_{prop}$ are the transmission and
  propagation delays of a single packet, respectively:
  \begin{equation*}
    d_{trans} = \frac{L}{R} \quad \mathrm{and} \quad d_{prop} = \frac{l}{s}
  \end{equation*}
  It gives:
  \begin{equation*}
    d = 12\frac{65,000}{40.10^9}+\frac{4,600.10^3}{2.5.10^8}
  \end{equation*}
  \begin{equation*}
    d = 0.0184s \quad (d= 18.4ms)
  \end{equation*}
}

\newpage
%% \question{What is the value of N beyond which the transmission delay is higher than the propagation delay?}

%% \answer{
%%   \begin{equation*}
%%   N\frac{L}{R} \geq \frac{d}{s} \\
%%   N \geq \frac{d.R}{s.L}
%%   \end{equation*}

%%   It gives:
%%   \begin{equation*}
%%     N \geq \frac{4,600.10^3.40.10^9}{2.5.10^8.65,000}\\
%%     N \geq 11,323
%%   \end{equation*}
%% }

\question{What are the five layers in the Internet protocol stack?
  What are the principal responsibilities of each of these layers?}
\answer{See slide 60 in Chapter 1.}

\newpage

\question{Consider a network link shared by N=11 users, each user
  being active 15\% of the time (the probability that a given user is
  active at any given time is $p=0.15$). What is the probability $p_2$ that exactly $2$
  users are active simultaneously?}

\answer{See slide 30 in Chapter 1.
  The probability that exactly $i$ users are active simultaneously is:
  \begin{equation*}
    p_i = \binom{N}{i} p^i(1-p)^{(N-i)}
  \end{equation*}
  It gives:
  \begin{equation*}
    p_2 = 55.0.15^2.(1-0.15)^{11-2}
  \end{equation*}
  \begin{equation*}
    p_2 = 0.29
  \end{equation*}
}

\newpage


%%%% Application Layer


\question{The content between the two lines below was captured using
  Wireshark:
  
\noindent \hrulefill\\
\begin{small}
\textbf{GET} / HTTP/1.1 \\
Host: www.concordia.ca\\
User-Agent: Mozilla/5.0 (X11; Fedora; Linux x86\_64; rv:51.0) Gecko/20100101 Firefox/51.0\\
Accept: text/html,application/xhtml+xml,application/xml;q=0.9,*/*;q=0.8\\
Accept-Language: en-US,en;q=0.5\\
Accept-Encoding: gzip, deflate\\
\textbf{Cookie}: \_\_utma=263452702.1080630157.1444061100.1487871467.1487874716.26;  [truncated]\\
Connection: keep-alive\\
Upgrade-Insecure-Requests: 1\\
\textbf{\textbackslash r\textbackslash n}\\
\vspace*{-0.3cm}
\hrulefill

\end{small}
\begin{enumerate}
\item To which protocol does this content correspond to?
\item To which layer in the Internet model does this protocol belong to?
\item What is the meaning of the strings in bold face (\textbf{GET}, \textbf{Cookie} and \textbf{\textbackslash r \textbackslash n})? 
\item Is this content coming from a client or from a server?
\item Describe a possible follow-up message for this content, i.e.,
  how the client or server may respond to this message. Only a
  high-level description of the message is expected, i.e., you don't
  have to write the complete message explicitly.
\end{enumerate}
}

\newpage

\answer{See slide 27 in Chapter 2. 
  \begin{enumerate}
  \item This trace is from the HTTP protocol.
  \item It belongs to the Application Layer.
  \item \textbf{GET}: HTTP request type (get an object). \textbf{Cookie}: a header field containing the value of the cookies previously set by the server for this client. \textbf{\textbackslash r\textbackslash n}: end of header.
  \item From a client, this is an HTTP request.
  \item The server will answer with an HTTP response that could contain the requested object (the / object in this case) or an error message.
  \end{enumerate}
}


% DNS

\question{What is the main purpose of Domain Name System (DNS)? Explain the
  difference between recursive and iterated DNS queries.}

\answer{See slide 62, 67 and 69 in Chapter 2. The main purpose of DNS is to resolve IP addresses from host
  names. DNS is a distributed database. In a recursive query, the
  query is forwarded from the client performing the query to the
  various servers involved in the response. In an iterated query, the
  client itself issues successive queries to the servers involved in the response.}

\newpage


%%%% Transport Layer

% UDP

\question{Explain when and how the field named ``source port'' in the UDP header might be used.}

\answer{This field is used when the destination of a UDP segment wants
  to send UDP segments back to the sender. In this case, the source
  port used by the sender becomes the destination port in the segments
  sent by the destination. The source port may also be used by a server to filter incoming packets.}

\newpage

% Reliable data transfer
% Flow control

\question{Explain the purpose, meaning and use of the receive window
  in TCP (\texttt{rwnd} variable).}

\answer{The receive window is used to implement \underline{flow control} in
TCP. It is set by the receiver in a header field of the segments sent
back to the sender. The receive window specifies the number of bytes
that the receive is willing to accept, i.e., \underline{the free space
  in the receiving buffer}. The sender uses this value to \underline{limit the
amount of in-flight bytes} (sent but non-acknowledged bytes).}

\newpage

% Congestion control

\question{Explain the difference between network-assisted and end-to-end
  congestion control. Which one of these approaches is used by TCP? And by UDP?}

\answer{See slide 95 in Chapter 3. In network-assisted congestion
  control, the routers are responsible for congestion detection and
  send feedback to the senders. In end-to-end congestion control, the
  senders themselves infer congestion status based on packet losses
  and delays, without receiving any explicit feedback from the
  network. TCP adopts an end-to-end approach to congestion
  control. UDP doesn't have any congestion control mechanism.}

\newpage

% TCP


\question{Two non-duplicate ACKs are received while a TCP sender is in Slow Start mode with \texttt{cwnd}=1KB, \texttt{ssthresh}=64KB and \texttt{MSS}=1KB. Describe the state of the TCP sender after reception of each of these two ACKs, more precisely:
  \begin{itemize}
  \item The state in which the TCP sender is (e.g., Slow Start).
  \item The value of the \texttt{cwnd} variable.
  \item The value of the \texttt{ssthresh} variable.
  \end{itemize}
}

\answer{See slide 104 in Chapter 3. \texttt{cwnd} is increased by 1 MSS after each ACK is received. \texttt{ssthresh} doesn't change. TCP remains in Slow Start mode.}

\end{document}
